\documentclass[12pt,a4paper]{article}
% Essential packages
\usepackage[utf8]{inputenc} % Character encoding
\usepackage{graphicx} % For images
\usepackage{amsmath} % For mathematical formulas
\usepackage{hyperref} % For hyperlinks
% Listings for code
\usepackage{listings}
\lstset{
 language=C,
 basicstyle=\ttfamily\small,
 numbers=left,
 frame=single,
 breaklines=true,
}
% Tighten space before sections
\usepackage{titlesec}
\titlespacing*{\section}{0pt}{0.5em}{0.5em}
\titlespacing*{\subsection}{0pt}{0.3em}{0.3em}
\setlength{\parindent}{0pt}

% Document metadata
\title{Documentation Group Project 1 (Assignment06)}
\author{Mario Neuner, Leander Decristoforo}
\date{\today}
\begin{document}
\maketitle
\tableofcontents
\pagebreak

\section{Problem01: 7001st prime}
\vspace{0pt}
\subsection{Implementation}
For this problem we thought of a simple looping strategy: start at 2 and loop through every number. While doing so check for
primes and increment a counter for every prime found. When the 7001st prime is found the while loop terminates and the last prime
(therefore the 7001st) is assigned to the result variable and printed out at the end of the file. Other variables used are:
\begin{itemize}
\item \texttt{validation} -- for confirming if a number is prime
\item \texttt{count} -- for counting the number of found primes
\item \texttt{run} -- for running through the natural numbers and checking each for primality
\end{itemize}
\subsection{Output}
\texttt{The program compiles with "gcc -Wall -Wextra -Werror -Wpedantic -std=c18 prime\_7001.c -o prime".}
When running \texttt{./prime} you get the output: 
\newline
"The 7001st prime is 70663.000000"

% Code listing with no float parameter
\begin{lstlisting}[caption={7001st prime in C},label={lst:p7001}]
 #include <stdio.h>
 #define NUM 7001
 // Authors: Mario Neuner, Leander Decristoforo
 int main(void) {
 int count = 0, validation = 0; // 0 means prime, 1 not
 long double result = 0, run = 2;
 while (count != NUM) {
 for (int i = 2; (i * i) <= run; i++) {
 // Check if prime
 if (((long long)run % i) == 0) {
 validation = 1;
 }
 }
 if (validation == 0) {
 count++;
 result = run;
 }
 run++;
 validation = 0;
 }
 printf("The 7001st prime is %Lf\n", result);
 return 0;
 }
\end{lstlisting}

\vspace{10em}

\section{Problem02: 5-6-numbers}
\subsection{Implementation}
For this problem we started a while loop that breaks when the desired numbers are found.
For this to happen we start with a few variables for keeping track of everything, you find 
the exact usage at the end of this section. Now we startet with n = 165 and m = 144 since we
have to find new values where sm = fn. For that we first compute sm and the start the while 
loop, in which we compute the current fn value and then again loop through every sm value until
the first element of sm being >= fn. Then we check whether sm is equal to fn and in that case we
found our new values. Otherwise we proceed with incrementing n and starting the loop all over
again. This takes a while, but after having found the new elements the code prints the new numbers.
The variables used are: 
\begin{itemize}
    \item \texttt{n} -- for tracking the n values
    \item \texttt{m} -- for tracking the m values
    \item \texttt{sm} -- for tracking the current sm sequence element
    \item \texttt{fn} -- for tracking the current fn sequence element
\end{itemize}
\subsection{Output}
\texttt{The program compiles with "gcc -Wall -Wextra -Werror -Wpedantic -std=c18 5\_6\_nums.c -o nums".}
When running \texttt{./nums} you get the output: 
\newline
"The next same number are at m = 27693, n = 31977 with fn = sm = 1533776805.000000 (1533776805.000000)"

\begin{lstlisting}[caption={7001st prime in C},label={lst:p7001}]
    #include <stdio.h>
    #include <stdlib.h>
    // Authors: Mario Neuner, Leander Decristoforo

    int main(void) {
        // Start at n 166 and m 144
        long n = 165, m = 144;       // 0 == not found, 1 == found
        long double sm = m * (2 * m -1), fn = 0;
        while (1) {
            // Calculate current sequence values
            fn = (n * (3 * n - 1)) / 2;
            // Go through elements of sm until >= fn
            while (fn > sm) {
                m++;
                sm = m * (2 * m -1);
            }
            // Check if sm equal to fn
            if (sm == fn) {
                break;
            }
            n++;
        }
        printf("The next same number are at m = %ld, n = %ld with fn = sm = %Lf (%Lf)\n", m, n, sm, fn);
        return EXIT_SUCCESS; 
    }
\end{lstlisting}

\vspace{10em}
   
\section{Problem03: primeseries-generator}
\subsection{Implementation}
Now this problem to be honest was a bit more challenging. For this problem we implemented a first function called "is\_prime"
for determining whether a given number is prime like we did in problems before. 
\newline
The second function called largest\_n finds for 
provided integers a, b and c the maximum value of n such that every natural number from 0 to n the formula computes a prime. We 
did this with a while loop that computes the formula and tests the result with the "is\_prime" function and for the case that the
result is less than 1 and thus not prime.  
\newline
In the main function we just looped through every possible values for a, b and c and first assured that c is prime, for the case 
n = 0. Then we proceeded by computing the largest value for n with the current a, b and c values via the "largest\_n" fuction and
afterwards compared it to maximum n value. In this way we can be sure to get the correct maximum n value. The program prints the 
max n value at the end of the program.
\newline
The variables used are:
\vspace{1em}
\newline
Main:
\begin{itemize}
    \item \texttt{max} -- for tracking the maximum n value
    \item \texttt{current\_n} -- for holding the current n value
    \item \texttt{a,b,c} -- for looping and getting the largest n
\end{itemize}

\vspace{1em}
larges\_n:
\begin{itemize}
    \item \texttt{nmax} -- for tracking the maximum n value
    \item \texttt{result} -- for storing the result of the formula computations
\end{itemize}
\subsection{Output}
\texttt{The program compiles with "gcc -Wall -Wextra -Werror -Wpedantic -std=c18 primeseriesg.c -o primeseries".}
When running \texttt{./primeseries} you get the output: 
\newline
"The maximum value for n is 28"

\vspace{3em}
\begin{lstlisting}[caption={7001st prime in C},label={lst:p7001}]
    #include <stdio.h>
    #include <stdlib.h>
    #define LIMIT 333
    
    // Checks prime, 0 if, 1 if not
    int is_prime(int n) {
        if (n < 2) {
            return 0;
        }
        for (int i = 2; i * i <= n; i++) {
            if (n % i == 0) {
                return 0;
            }
        }
        return 1;
    }
    
    // Check for largest n with given a,b,c
    long long largest_n(int a, int b, int c) {
        long long nmax = 0;
        long long result = 0;
        while (1) {
            result = (nmax * nmax * nmax) + a * (nmax * nmax) + b * nmax + c;
            if (!is_prime(result) || result < 2) {
                return nmax - 1;
            }
            nmax++;
        }
    }
    
    int main(void) {
        long long max = 0, current_n = 0;
    
        // Loop through every possible a,b and c
        for (int a = -LIMIT + 1; a < LIMIT; a++) {
            for (int b = -LIMIT; b < LIMIT; b++) {
                for (int c = -LIMIT; c < LIMIT; c++) {
                    if(!is_prime(c)) {
                        continue;
                    }
                    // Compute largest n for a,b,c
                    current_n = largest_n(a, b, c);
                    if (current_n > max) {
                        max = current_n;
                    }
                }
            }
        }
        fprintf(stdout, "The maximum value for n is %lld\n", max);
        return EXIT_SUCCESS;
    }
\end{lstlisting}

\end{document}